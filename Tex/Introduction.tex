\section{引言}
\par{随着技术的不断进步,人类社会对能源的需求量日益俱增,传统的化石能源行业的压力也愈来愈大,以太阳能、生物质能为代表的新能源应运而生,受到越来越多能源工作者的关注。而随着餐饮业的快速发展,我国每年因餐饮产生的废弃油脂可达500 万吨\cite{2014我国餐厨废油资源化利用现状及展望,马冰2015油脂催化转化为绿色燃料的技术进展},废弃油脂已经成为城市环境污染的主要源头之一,所以对废弃油脂的高效利用成为当前关注的重点。油脂的主要成分是脂肪酸酯,利用催化裂化工艺可以加工不同来源、不同性质的各种脂肪酸酯原料,通过调整催化剂和操作条件,可以实现以脂肪酸酯原料生产重要化工品和清洁燃料的目的\cite{脂肪酸酯的催化裂化研究}。由于H-ZSM-5 催化剂具有水热稳定性能好,择形选择性高(其中芳烃的选择性尤为高)等优点,已被广泛用于脂肪酸酯的催化裂解工艺中。在脂肪酸酯的催化裂解过程中,脂肪酸酯在分子筛中的吸附和扩散性质对催化裂解的产物分布和选择性有重要影响\cite{闫昊2017基于结构导向集总的废弃油脂催化裂化分子尺度动力学模型},是影响催化裂化反应的重要前驱因素。}
\par{目前通过传统的实验方法来获得吸附性质的效率较低,尤其是吸附位和吸附构型的确定,而在反应条件下研究长链分子的吸附更为困难。分子模拟技术是近十几年发展起来的新技术,可以模拟分子运动的微观行为,很大程度上克服了不能在极端条件下进行实验的缺陷,并且可以大幅度缩短实验时间,降低实验成本。众多学者也将分子模拟技术用于解释和预测分子筛催化剂上反应物的吸附研究:郭相丹\cite{2006Investigation} 等人分别采用智能重量分析仪和巨正则系综蒙特卡洛方法,研究了H-ZSM-5 分子筛对水的吸附,发现模拟结果与实验结果相当吻合;李健博\cite{几种气体在ZSM-5分子筛上吸附的模拟与实验研究}分别采用分子模拟和实验的手段,研究了小分子气体在ZSM-5分子筛上的吸附情况,研究结果表明分子模拟得到的吸附量随压力的变化规律与实验一致。}
\par{虽然许多研究者已经利用实验测定、理论计算和分子模拟的方法对流体在H-ZSM-5 分子筛上的吸附和扩散行为进行了研究,但针对应用于具体催化裂解反应体系中的具有不同孔径大小的H-ZSM-5 分子筛催化剂上脂肪酸甲酯的吸附和扩散研究目前尚缺少,此方面的研究对于了解脂肪酸甲酯在H-ZSM-5 分子筛上的吸附和扩散机理以及脂肪酸甲酯催化反应的实质将具有重要的理论及实际意义。}
\par{本论文采用分子模拟的手段,通过计算脂肪酸甲酯模型化合物在微介孔H-ZSM-5 分子筛内的吸附量、吸附等温线、吸附热、吸附位等信息,来分析模型化合物不饱和度、链长以及分子筛孔径大小对吸附扩散的影响,从而为脂肪酸甲酯的催化裂解过程中催化剂的设计提供理论指导。}
