\section{结论}
\par{本文使用Materials Studio 8.0软件对脂肪酸甲酯在微介孔H-ZSM-5分子筛上的吸附扩散进行模拟,通过分析模型化合物的吸附扩散性能得到以下结论:}
\begin{enumerate}
    \item 丁酸甲酯等模型化合物在微介孔H-ZSM-5分子筛的吸附等温线均属于\RNum{1}型等温线。随着孔径的增大,饱和吸附量减小,吸附平衡常数减小,单位质量分子筛的饱和吸附量先增大后减小,20Å介孔的最大;随着温度的增大,饱和吸附量减小,吸附平衡常数减小;随着吸附分子极性增强,饱和吸附量增大,吸附平衡常数增大;随着吸附分子链长的增大,饱和吸附量减小,吸附平衡常数增大。
    \item 丁酸甲酯在微孔和介孔中的等量吸附热随压力(吸附量)的变化不同:在微孔中,随着压力的增大,等量吸附热以增量逐渐减小的方式增大,最后趋于最大值,而随着吸附量的增大,等量吸附热线性增大;在介孔中,随着压力或吸附量的增大,等量吸附热先增大再减小,有一个极大值。随着孔径的增大,相同压力下的等量吸附热减小。随着温度的增加,微孔介孔的等量吸附热都减小。
    \item 丁酸甲酯优先吸附在活性吸附位(B位酸)处,并分布在交叉孔道旁,B位酸与双键氧形成氢键,随着压力的增大,丁酸甲酯在交叉孔道和直孔道都有吸附。而丁烯酸甲酯中的双键与B位酸形成 π 配位超分子复合物。
    \item 丁酸甲酯扩散系数的大小顺序为60Å介孔> 20Å介孔$\gg$微孔,这个结果与实验结果一致。但随着孔径的增大,孔径大小对扩散系数的影响变小:从微孔到20Å介孔,扩散系数增加了89.6倍;而从20Å介孔到60Å介孔,扩散系数只增加了1.4倍。
\end{enumerate}
\par{介孔H-ZSM-5分子筛相比于微孔H-ZSM-5分子筛,单位质量下对脂肪酸甲酯的饱和吸附量增大,等量吸附热减小,扩散速率显著提高,吸附扩散性能增强。但介孔的孔径并不是越大越好,在单位质量的20Å介孔比在60Å介孔的饱和吸附量大,在60Å介孔中的扩散系数只比在20Å介孔的增加了1.8倍。}
